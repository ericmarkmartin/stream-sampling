% \usepackage{exam}
\documentclass{article}
\usepackage[utf8]{inputenc}
\usepackage[letterpaper, margin=1in]{geometry}
\usepackage{amsmath}
\usepackage{amsfonts}
\usepackage[shortlabels]{enumitem}
\usepackage{librebaskerville}
\usepackage{setspace}
\usepackage[hang]{footmisc}
\usepackage{titling}
\usepackage{atbegshi}
\usepackage{fancyhdr}

\pagestyle{fancy}
\fancyhf{}
\renewcommand{\headrulewidth}{0pt}
\rhead{AMS Eric Martin}

\pretitle{\begin{flushleft}\vspace{-5em}\LARGE}
\posttitle{\vspace{-3em}\end{flushleft}}

\title{Stream Sampling}
\author{}
\date{}

\setlength{\footnotemargin}{2mm}

\begin{document}

\maketitle
\thispagestyle{fancy}

% \linespread{1.75}
\onehalfspacing

\begin{enumerate}[1.]
\item %1
  I'm going to start listing numbers off to you. I'll stop at some arbitrary
  point and ask you to give me a uniformly sampled number from the ones I
  listed.
  \begin{enumerate}[a.]
  \item %a
    Can you design a simple algorithm to do this?
  \item %b
    Now what if I ask you to give me a uniform sample of $k$ numbers without
    replacement from the ones I gave (I'll tell you $k$ up front). Does your
    solution easily adapt?

  \item %c
    Let's think about our solution and get a sense of how efficient it is. In
    computer science, we measure efficiency in few different metrics. Two of the
    most important are \textit{time} and \textit{space}.

    \smallskip

    \textit{Time} is a question of how many steps an algorithm needs to
    compute an output. \textit{Space} is a question of how much information an
    algorithm needs to ``write down'' while computing this ou.

    \smallskip

    Specifically, we care about the \textit{complexity} of these resources, which
    is a measurement of how much of the resource in question an algorithm uses
    as a function of the input size.

    \smallskip

    Consider as an example an algorithm that looks for a target number in a list
    of $n$ numbers. In the worst case, it'll have to look at every number in the
    list---of which there are $n$---once, so it takes $n$ steps. We say this
    algorithm\footnote{which is called linear search} is $O(n)$, or linear.
    \smallskip

    On the other hand, the algorithm only has to write down the thing it's
    looking for. Assuming we can write a number down in a constant amount of
    space\footnote{In particularly strict theoretical analyses, this doesn't
      quite hold but it's quite accurate to real-world computing.}, we say it
    takes $O(1)$, or ``constant'', space.
    
    \smallskip

    We usually consider algorithms time-efficient and/or space-efficient if they
    use $O(n)$ time and/or $O(\log n)$ space respectively. Suppose I ultimately
    list off $n$ numbers. Does our algorithm meet these standards?
  \end{enumerate}

\clearpage
\item %2
  Let's think about why we use as much space as we do and how we can use less.
  
  \begin{enumerate}[a.]
  \item %a
    Let's simplify the problem for a moment by only sampling a single
    number---i.e. fixing $k=1$.

    If I ask you to give me the minimum number I list instead of a random one,
    can you come up with a more space-efficient solution?

  \item %b
    Can we turn the random-sampling problem into a min-finding problem?

    \smallskip

    \textit{Hint: instead of just storing the numbers as they come in, what if
      we also store a random ``tag'' sampled uniformly from} $[0, 1)$
    \textit{? You may assume that we can take such samples at will.}

  \item %c
    Can we scale this solution into one that works for a sample of $k$ numbers
    (again without replacement)? What are the time and space costs of this
    solution?

    For time, you should note that it is possible to maintain a collection of
    $k$ numbers using $O(k)$ space\footnote{I'm making the simplifying
      assumption here that it takes $O(1)$---that is, constant---space to write
      down a number.} such that it costs:
    \begin{itemize}
    \item $O(1)$ time to find the maximum
    \item $O(\log k)$ time to remove the maximum
    \item $O(\log k)$ time to add a new item to the collection.
      \footnote{This can be done by arranging the $k$ elements into a data structure called a binary heap.}
    \end{itemize}

  \item %d
    Suppose that while I'm listing numbers, I can stop you multiple times to get
    up-to-then random samples. Is there a statistical difference bewteen the
    samples given by solutions 1 and 2?
  \end{enumerate}

\clearpage
\item %3
  Can we do even better?

  \begin{enumerate}[a.]
  \item %a
    Let's again go back to the $k=1$ case. Let $E_i$ denote the event in which
    we take the $i$th number as our current sample. We'll always take the first number we
    are given as the sample, so $\mathbb{P}[E_1] = 1$. What is
    $\mathbb{P}[E_2]$? $\mathbb{P}[E_1]$?

  \item %b
    How can we use this idea to make a new version of our sampler? What are the
    time and space efficiency of this new version?

  \item %c
    Can you prove that this solution gives a uniform random sample? I.e. let $X$
    be a random variable for the output of the sampler. Can you prove that for
    every number $x$ in a list of $n$ numbers, $\mathbb{P}[X = x] =
    \frac{1}{n}$? Assume that every number in the list is distinct.

    \textit{Hint: try induction on $i$, the number of numbers we've run through
      the sampler.}

  \item %d
    Can we generalize this solution to work for a size $k$ sample?

  \item %e
    This final solution is called \textit{reservoir sampling}, and is deployed
    in the real world in order to take samples from massive real-time data sets.
    What are it's time and space costs?

  \end{enumerate}
\end{enumerate}

\end{document}
